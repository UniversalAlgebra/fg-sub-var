\begin{filecontents*}{inputs/refs.bib}
@article {MR3239624,
    AUTHOR = {Valeriote, M. and Willard, R.},
     TITLE = {Idempotent {$n$}-permutable varieties},
   JOURNAL = {Bull. Lond. Math. Soc.},
  FJOURNAL = {Bulletin of the London Mathematical Society},
    VOLUME = {46},
      YEAR = {2014},
    NUMBER = {4},
     PAGES = {870--880},
      ISSN = {0024-6093},
   MRCLASS = {08A05 (06F99 68Q25)},
  MRNUMBER = {3239624},
       DOI = {10.1112/blms/bdu044},
       URL = {http://dx.doi.org/10.1112/blms/bdu044},
}
\end{filecontents*}
\documentclass[11pt]{amsart}
% The following \documentclass options may be useful:
% preprint      Remove this option only once the paper is in final form.
% 10pt          To set in 10-point type instead of 9-point.
% 11pt          To set in 11-point type instead of 9-point.
% numbers       To obtain numeric citation style instead of author/year.

%% \usepackage{setspace} %\onehalfspacing

\usepackage{amsmath}
\usepackage{amscd,amssymb,amsthm} %, amsmath are included by default
\usepackage{latexsym,stmaryrd,mathrsfs,enumerate,scalefnt,ifthen}
\usepackage{mathtools}
\usepackage[mathcal]{euscript}
\usepackage[colorlinks=true,urlcolor=black,linkcolor=black,citecolor=black]{hyperref}
\usepackage{url}
\usepackage{scalefnt}
\usepackage{tikz}
\usepackage{color}
\usepackage[margin=1in]{geometry}
\usepackage{scrextend}
\usepackage{multirow}

%%////////////////////////////////////////////////////////////////////////////////
%% Theorem styles
\numberwithin{equation}{section}
\theoremstyle{plain}
\newtheorem{theorem}{Theorem}[section]
\newtheorem{lemma}[theorem]{Lemma}
\newtheorem{proposition}[theorem]{Proposition}
\newtheorem{prop}[theorem]{Proposition}
\theoremstyle{definition}
\newtheorem{claim}[theorem]{Claim}
\newtheorem{corollary}[theorem]{Corollary}
\newtheorem{definition}[theorem]{Definition}
\newtheorem{notation}[theorem]{Notation}
\newtheorem{Fact}[theorem]{Fact}
\newtheorem*{fact}{Fact}
\newtheorem{example}[theorem]{Example}
\newtheorem{examples}[theorem]{Examples}
\newtheorem{exercise}{Exercise}
\newtheorem*{lem}{Lemma}
\newtheorem*{cor}{Corollary}
\newtheorem*{remark}{Remark}
\newtheorem*{remarks}{Remarks}
\newtheorem*{obs}{Observation}


%%%%%%%%%%%%%%%%%%%%%%%%%%%%%%%%%%%%%%%%
% Acronyms
%%%%%%%%%%%%%%%%%%%%%%%%%%%%%%%%%%%%%%%%
%% \usepackage[acronym, shortcuts]{glossaries}
%\usepackage[smaller]{acro}
\usepackage[smaller]{acronym}
\usepackage{xspace}

%% \acs{CSP} -- short version of the acronym\\
%% \acl{CSP} -- expanded acronym without mentioning the acronym.\\
%% \acp{CSP} -- plurals.\\
%% \acfp{CSP} -- long forms into plurals.\\
%% \acsp{CSP} -- short form into a plural.\\
%% \aclp{CSP} -- long form into a plural.\\
%% \acfi{CSP} -- Full Name acronym in italics and abbreviated form in upshape.\\
%% \acsu{CSP} -- short form of the acronym and marks it as used.\\
%% \aclu{CSP} -- Prints the long form of the acronym and marks it as used.\\

\acrodef{lics}[LICS]{Logic in Computer Science}
\acrodef{sat}[SAT]{satisfiability}
\acrodef{nae}[NAE]{not-all-equal}
\acrodef{ctb}[CTB]{cube term blocker}
\acrodef{tct}[TCT]{tame congruence theory}
\acrodef{wnu}[WNU]{weak near-unanimity}
\acrodef{CSP}[CSP]{constraint satisfaction problem}
\acrodef{MAS}[MAS]{minimal absorbing subuniverse}
\acrodef{MA}[MA]{minimal absorbing}
\acrodef{cib}[CIB]{commutative idempotent binar}
\acrodef{sd}[SD]{semidistributive}
\acrodef{NP}[NP]{nondeterministic polynomial time}
\acrodef{P}[P]{polynomial time}
\acrodef{PeqNP}[P $ = $ NP]{P is NP}
\acrodef{PneqNP}[P $ \neq $ NP]{P is not NP}

%%% This enables markdown style bold and italic. %%%
\makeatletter
\def\starparse{\@ifnextchar*{\bfstarx}{\itstar}}
\def\bfstarx#1{\@ifnextchar*{\bfitstar\@gobble}{\bfstar}}
\makeatother
\def\itstar#1*{\textit{#1}\starON}
\def\bfstar#1**{\textbf{#1}\starON}
\def\bfitstar#1***{\textbf{\textit{#1}}\starON}
\def\starON{\catcode`\*=\active}
\def\starOFF{\catcode`\*=12}
\starON
\def*{\starOFF \starparse}
\starOFF

%%%%%%%%%%%%%%%%%%%%%%%%%%%%%%%%%%%%%%%%%%%%%%%%%%%%%%%%%%%%%%%%%

\usepackage{inputs/proof-dashed}


%%%%%%%%%%%%%%%%%%%%%%%%%%%%%%%%%%%%%%%%%%%%%%%%%%%%%%%%%%%%%%%%%

%% Put new macros in the macros.sty file
\usepackage{inputs/macros}

% \usepackage[backend=bibtex]{biblatex}
% \bibliography{inputs/refs.bib}

\begin{document}
\starON
\title[Notes on n-permutability]{Notes on n-permutability}
\date{\today}
\author[W.~DeMeo and P.~Mayr]{William DeMeo and Peter Mayr}
\address{University of Colorado}
\email{williamdemeo@gmail.com}

\section{Introduction}
\begin{example}
  \label{four-elt-grpd}
Let $\bA_4 = \<\{0,1,2,3\}, \circ\>$ be the idempotent semigroup with binary operation $\circ$ given by the table below.
\vskip3mm
 \begin{center}
 \begin{tabular}{c|cccc}
      $\circ$ & 0 & 1 & 2 & 3\\
      \hline
      0 & 0 & 2 & 1 & 1\\
      1 & 0 & 1 & 3 & 2\\
      2 & 0 & 3 &2 & 1\\
      3 & 1 & 2 & 1 & 3
    \end{tabular}
 \end{center}
\vskip3mm
%% \thanks{The authors would li}

The variety $\mathbb{V}(\mathbf A)$ generated by $\mathbf A$ is CM, not CD, and has no edge term ($(\{1,2,3\},  \{0,1,2,3\})$ is a cube term blocker).
There are also six-element algebras with these properties. Take, for example, the algebra $\bA_6 = \<\{0,1,2,3,4,5\}, \circ\>$ with operation table shown below.
\vskip3mm
 \begin{center}
 \begin{tabular}{c|cccccc}
      $\circ$ & 0 & 1 & 2 & 3 & 4 & 5\\
      \hline
      0 & 0 & 2 & 1 & 1 & 1 & 1\\
      1 & 0 & 1 & 3 & 5 & 2 & 4\\
      2 & 0 & 3 & 2 & 4 & 5 & 1\\
      3 & 0 & 5 & 4 & 3 & 1 & 2\\
      4 & 0 & 2 & 5 & 1 & 4 & 3\\
      5 & 1 & 4 & 1 & 2 & 3 & 5
    \end{tabular}
 \end{center}
\vskip3mm

More generally, the following table yields a $2n$-element algebra with the same properties:
\vskip3mm
 \begin{center}
 \begin{tabular}{c|cccccc}
      $\circ$ & 0 & 1 & 2 & 3 & $\cdots$ & $2n-1$\\
      \hline
            0 & 0      & 2 & 1 & 1 & $\cdots$ & 1\\
            1 & 0      &   &   &   &        & \\
            2 & 0      &   &   &   &        & \\
       $\vdots$ & $\vdots$ &  && \multicolumn{3}{c}{$\mathrm{Squag}_{2n-1}$}          \\
         $2n-2$ & 0      &   &   &   &        & \\
         $2n-1$ & 1      &   &   &   &        & 
    \end{tabular}
 \end{center}


\vskip3mm
Specializing to $3$-permutability for now, a \emph{local Hagemann-Mitchke} sequence for the set 
\[\{(a_0, b_0, 0), (a_1, b_1, 1) , (a_2, b_2, 2) \} \subseteq A \times A \times \{0,1,2\}\] is a triple, $\mathbf p = (p_0, p_1, p_2)$, of 3-place terms satisfying
\begin{align}
a_0 &= p_0(a_0, a_0, b_0) =p_1(a_0, b_0, b_0) \nonumber\\
    &= p_1(a_1, a_1, b_1) =p_2(a_1, b_1, b_1) \nonumber\\
    &= p_2(a_2, a_2, b_2) =p_3(a_2, b_2, b_2) = b_2 \nonumber
\end{align}
Where we take $p_0$ and $p_3$ to be the first and third projections, respectively.

\end{example}

More generally, for $0\leq i < m$ and
$(a_i, b_i, i)\in A \times A \times \{0,1,\dots,m-1\}$,
we call $p = (p_0, p_1, \dots, p_m)$ a \emph{local Hagemann-Mitchke} sequence for 
$(a_i, b_i, i)$ if $p_i(a_i, a_i, b_i) = p_{i+1}(a_i, b_i, b_i)$, where $p_0, p_m$ are the first and third projections, respectively.

Valeriote and Willard prove in~\cite{MR3239624} that an idempotent algebra $\mathbf{A}$ generates a congruence $(n+1)$-permutable variety if and only if every subset
of $A \times A \times \{0, 1, \dots, n\}$ has a local Hagemann-Mitchke sequence.

\newpage

\noindent {\bf Question:} Can the pair of terms below be used as the (non-projection) terms in a local Hagemann-Mitchke sequence $\mathbf p = (p_0, p_1, p_2, p_3)$ for the examples above.
\begin{align}
p_1(x,y,z) &= (x (y z) )x\nonumber\\
p_2(x,y,z) &= (x (x (y z)) )x\nonumber
\end{align}

\noindent {\bf Answer:} No, these don't work. (insert counterexample)

Here are some other tables that also yield algebras generating a CM, non-distributive variety with no cube term:

\begin{example}
  \label{four-elt-grpd}
Consider the idempotent semigroup 
$\bA_i = \<\{0,1,2,3\}, \circ_i\>$ with binary operation $\circ_i$ given in the $i$-th table below.
\vskip3mm
 \begin{center}
 \begin{tabular}{c|cccc}
      $\circ_3$ & 0 & 1 & 2 & 3\\
      \hline
           0    & 0 & 3 & 2 & 1\\
           1    & 0 & 1 & 3 & 2\\
           2    & 3 & 3 & 2 & 1\\
           3    & 3 & 2 & 1 & 3
    \end{tabular}\hskip1cm
 \begin{tabular}{c|cccc}
      $\circ_2$ & 0 & 1 & 2 & 3\\
      \hline
           0    & 0 & 2 & 3 & 1\\
           1    & 0 & 1 & 3 & 2\\
           2    & 2 & 3 & 2 & 1\\
           3    & 2 & 2 & 1 & 3
    \end{tabular}\hskip1cm
 \begin{tabular}{c|cccc}
      $\circ_2$ & 0 & 1 & 2 & 3\\
      \hline
           0    & 0 & 2 & 1 & 3\\
           1    & 0 & 1 & 3 & 2\\
           2    & 1 & 3 & 2 & 1\\
           3    & 1 & 2 & 1 & 3
    \end{tabular}
 \end{center}
 \vskip5mm
 \begin{center}
 \begin{tabular}{c|cccc}
      $\circ_2$ & 0 & 1 & 2 & 3\\
      \hline
           0    & 0 & 2 & 1 & 1\\
           1    & 0 & 1 & 3 & 2\\
           2    & 0 & 3 & 2 & 1\\
           3    & 1 & 2 & 1 & 3
    \end{tabular}\hskip1cm
 \begin{tabular}{c|cccc}
      $\circ_2$ & 0 & 1 & 2 & 3\\
      \hline
           0    & 0 & 3 & 1 & 1\\
           1    & 0 & 1 & 3 & 2\\
           2    & 0 & 3 & 2 & 1\\
           3    & 1 & 2 & 1 & 3
    \end{tabular}\hskip1cm
 \begin{tabular}{c|cccc}
      $\circ_3$ & 0 & 1 & 2 & 3\\
      \hline
           0    & 0 & 2 & 1 & 1\\
           1    & 0 & 1 & 3 & 2\\
           2    & 0 & 3 & 2 & 1\\
           3    & 1 & 2 & 1 & 3
    \end{tabular}
 \end{center}
\end{example}



\bibliographystyle{alphaurl}
\bibliography{inputs/refs} 

\end{document}
